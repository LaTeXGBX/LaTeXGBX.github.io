% Attention à l’ indentation du texter , cela permet de relire plussimplement en cas de debuggage
\begin{equation}
  L _{\Omega ,\nu }^{\circ } (\nu ,T ) =\frac{2\mathrm { h } \nu ^{3}}{
    c ^{2}} \frac{1}{\mathrm { e } ^{\mathrm { h } \nu /(\mathrm { k }
    T ) } -1}
  \label{ eq : Planck } % On étiquette l’équation
\end{equation}
% On la cite dans le texte ...
J ’ aime beaucoup l ’ équation de Planck \ref{ eq : Planck }
%Voilà comment  on  introduit  un système d’équations  avec  align
\begin{align}
2x - 5y &=   8 \\
3x + 9y &=   -12
\end{align}
